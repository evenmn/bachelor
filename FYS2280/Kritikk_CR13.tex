\documentclass[norsk,a4paper,12pt]{article}
\usepackage{booktabs}  % Tabs
\usepackage{graphicx}  % Pictures/figures
\usepackage{listings}  % Source code
\usepackage{color}     % Colors
\usepackage{mdframed}  % Frames
\usepackage{float}
\usepackage{hyperref}
\usepackage{subcaption}

\begin{document}
\begin{center}
\includegraphics[width=0.15\textwidth]{uio.png}\par\vspace{1cm}
{\scshape\LARGE FYS2280 - Romteknologi \par}
\vspace{0.5cm}
{\scshape\large CaNoRock XIII\par}
\vspace{1cm}
{\Large\itshape Even Marius Nordhagen\par}
\vspace{0.5cm}
{\large \today\par}
\end{center}
\section*{Tanker om CaNoRock 13}
N{\aa}r jeg f{\o}rst fikk h{\o}re om CaNoRock, synes jeg med en gang det hele h{\o}rtes kult ut, s{\aa} jeg n{\o}lte ikke med {\aa} s{\o}ke. Dette gjorde at jeg umiddelbart fikk ganske h{\o}yre forventinger til opplegget, og informasjonsm{\o}tet f{\o}r turen gjorde ikke forventningene lavere. Det eneste jeg var litt nerv{\o}s for var det {\aa} snakke engelsk, det er ikke noe jeg gj{\o}r s{\aa} ofte, s{\aa} ferdighetene er deretter.\par
Noe av det f{\o}rste som m{\o}tte oss da vi kom til romsenteret, var 10 kanadere som sikkert var minst like spente p{\aa} oppholdet som oss. Det skulle fort vise seg at det {\aa} snakke engelsk ikke var noe {\aa} v{\ae}re nerv{\o}s for, de var t{\aa}lmodige og lette {\aa} forst{\aa}. \par
Videre ble vi introdusert for selve romsenteret, noe som var veldig kult. Jeg var fasinert av raketter f{\o}r jeg kom til And{\o}ya Space Center (ASC), og fasinasjonen ble ikke akkurat dempet i l{\o}pet av turen. 

\subsection*{Positivt}
Det kan vi vanskelig {\aa} ramse opp alt som var positivt, s{\aa} jeg vil ta med det viktigste. For det f{\o}rste er jeg veldig forn{\o}yd med hostelet. Standarden p{\aa} rommet var bra, og utstyrt med alt men trengte. Jeg var en av de f{\aa} studentene som fik tildelt enkeltrom, noe jeg ikke helt vet grunnen til, men det kunne nok kanskje v{\ae}rt litt mer sosialt {\aa} dele rom med noen andre. N{\aa}r det kommer til maten, har jeg ikke noe {\aa} utsette p{\aa} den. Det virket som de p{\aa} kj{\o}kkenet hadde peiling p{\aa} hva de drev med, og en stort pluss at de brukte restene. Creds til kj{\o}kkencrewet! Det var ikke tid til {\aa} kjede seg, ettersom det var opplegg fra morgen til kveld.

\subsection*{Negativt}
Jeg har egentlig ikke s{\aa} mye negativt {\aa} komme med. Det eneste jeg kommer p{\aa} er at jeg skulle {\o}nske simuleringen av raketten kunne simulert rakettens parametere i noe lengre tid enn de f{\o}rste 0.002 sekundene (da ville datene v{\ae}rt mer realistiske). Dette kunne v{\ae}rt gjort ved at vi startet simuleringen tidligere, men mest utslag ville det nok gjort om vi hadde brukt en kraftigere datamaskin. 

\subsection*{Konklusjon}
Alt i alt var dette en veldig vellykket tur, hvor kanskje det morsommeste var {\aa} m{\o}te nye folk fra andre land og dele erfaringer med dem. Ellers var det jo veldig g{\o}y {\aa} launche rakett, noe man ikke f{\aa}r gjort s{\aa} mange ganger i livet. Det var tydelig at det var en kompetent mannskap som sto bak oppholdet, og at opplegget var gjennomf{\o}rt f{\o}r.\par 
Jeg er veldig glad for at jeg ble med p{\aa} turen, og vil absolutt anbefale dette til likesinnede. Erfaringene kan komme godt med blant annet n{\aa}r jeg reiser til Singapore p{\aa} utveksling til v{\aa}ren. 
\end{document}