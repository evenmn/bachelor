\documentclass{scrartcl}
\usepackage{physics}

\title{FYS3110\\Quantum Mechanics}
\subtitle{Oblig 02}
\author{Even Marius Nordhagen}
\date{\today}

\begin{document}
\maketitle
\newpage
\section*{Introduction}
The purpose of this problem set is to be more familiar with braket notation and learn how to connect it with matrix notation. We will also see that we can solve some problems much faster with braket notation than with matrix notation, or at least make it much more transparent. 

\section*{Exercise 2.1}
In this exercise we are supposed to use a operator $\hat{H}$ defined by
$$\hat{H}=\mqty(1 & \frac{i}{2} & 0 \\ -\frac{i}{2} & 1 & 0 \\ 0 & 0 & \frac{1}{2})$$
\subsection*{a)}
To show that $\hat{H}$ is hermitian, I have to transpose and conjugate it and still have the same matrix.
$$\hat{H}^\dagger=\mqty(1 & -\frac{i}{2} & 0 \\ \frac{i}{2} & 1 & 0 \\ 0 & 0 & \frac{1}{2})^*=\mqty(1 & \frac{i}{2} & 0 \\ -\frac{i}{2} & 1 & 0 \\ 0 & 0 & \frac{1}{2})=\hat{H}$$
As you can see, the operator is hermitian.
\subsection*{b)}
If $\ket{i}$ for $i = 1,2,3$ are eigenkets of $\hat{H}$, they have to give a constant (an eigenvalue) and the same eigenket when they are multiplied with $\hat{H}$:
$$\hat{H}\ket{1}=\mqty(1 & \frac{i}{2} & 0 \\ -\frac{i}{2} & 1 & 0 \\ 0 & 0 & \frac{1}{2})\cdot\frac{1}{\sqrt{2}}\mqty(i \\ 1 \\ 0)=\frac{3}{2}\cdot\frac{1}{\sqrt{2}}\mqty(i \\ 1 \\ 0)$$
$$\hat{H}\ket{2}=\mqty(1 & \frac{i}{2} & 0 \\ -\frac{i}{2} & 1 & 0 \\ 0 & 0 & \frac{1}{2})\cdot\mqty(0 \\ 0 \\ 1)=\frac{1}{2}\mqty(0 \\ 0 \\ 1)$$
$$\hat{H}\ket{3}=\mqty(1 & \frac{i}{2} & 0 \\ -\frac{i}{2} & 1 & 0 \\ 0 & 0 & \frac{1}{2})\cdot\frac{1}{\sqrt{3}}\cdot\mqty(-i \\ 1 \\ -i)=\frac{1}{2}\cdot\frac{1}{\sqrt{3}}\mqty(-i \\ 1 \\ -1)$$
As we can see, the eigenvalues are $\frac{3}{2}$ and $\frac{1}{2}$ where the last one has multiplicity 2.

\subsection*{c)}
In this sub exercise I will find the eigenket basis that I used in the previous sub exercise. I have already calculated $\hat{H}\ket{i}$ for $i=1,2,3$, which is half the job. So let's get started!

$$\mel*{1}{\hat{H}}{1}=\frac{1}{\sqrt{2}}\mqty(-i & 1 & 0)\frac{3}{2}\cdot\frac{1}{\sqrt{2}}\mqty(i \\ 1  \\ 0) = \frac{3}{2}$$

$$\mel*{2}{\hat{H}}{1}=\mqty(0 & 0 & 1)\frac{3}{2}\cdot\frac{1}{\sqrt{2}}\mqty(i \\ 1 \\ 0) = 0$$

$$\mel*{3}{\hat{H}}{1}=\frac{1}{\sqrt{3}}\mqty(i & 1 & -1)\frac{3}{2}\cdot\frac{1}{\sqrt{2}}\mqty(i \\ 1 \\ 0) = 0$$

$$\mel*{1}{\hat{H}}{2}=\frac{1}{\sqrt{2}}\mqty(-i & 1 & 0)\cdot\frac{1}{2}\mqty(0 \\ 0 \\ 1) = 0$$

$$\mel*{2}{\hat{H}}{2}=\mqty(0 & 0 & 1)\cdot\frac{1}{2}\mqty(0 \\ 0 \\ 1)=\frac{1}{2}$$

$$\mel*{3}{\hat{H}}{2}=\frac{1}{\sqrt{3}}\mqty(i & 1 & -1)\cdot\frac{1}{2}\mqty(0 \\ 0 \\ 1)=-\frac{1}{2\sqrt{3}}$$

$$\mel*{1}{\hat{H}}{3}=\frac{1}{\sqrt{2}}\mqty(-i & 1 & 0)\cdot\frac{1}{2}\mqty(-i \\ 1 \\ -1) = 0$$

$$\mel*{2}{\hat{H}}{3}=\mqty(0 & 0 & 1)\cdot\frac{1}{2}\mqty(-i \\ 1 \\ -1)=-\frac{1}{2\sqrt{3}}$$

$$\mel*{3}{\hat{H}}{3}=\frac{1}{\sqrt{3}}\mqty(i & 1 & -1)\cdot\frac{1}{2}\mqty(-i \\ 1 \\ -1)=\frac{1}{2}$$
We end up with the matrix:
$$H_{ij}=\mqty(\frac{3}{2} & 0 & 0 \\ 0 & \frac{1}{2} & -\frac{1}{2\sqrt{3}} \\ 0 & -\frac{1}{2\sqrt{3}} & \frac{1}{2})$$
As you can see, this matrix is diagonal.
\subsection{d)}
Here I need to use Gram-Schmidt. The Gram-Schmidt process is a method used for orthonormalizing a set of vectors in an inner product space. It works as follows:
$$u_1=v_1$$
$$u_2=v_2-Proj_{u_1}(v_2)$$
$$u_3=v_3-Proj_{u_1}(v_3)-Proj_{u_3}(v_3)$$
Etc..
Where
$$Proj_{u_i}(v_j)=\frac{<v_j, u_i>}{<u_i, u_i>}\cdot u_i$$
In our special case, $v_1=\ket{1}$, $v_2=\ket{2}$ and $v_3=\ket{3}$\\\\
From the vectors in the introduction of the exercise, I notice that all the vectors are normalized, so the Kronecker-Delta tells us that the inner product between the same vector has to be equal to 1 ($\braket{v_i}{v_i}=1$). This makes the projection a little simpler. 
$$\ket{1'}=\ket{1}$$
$$\ket{2'}=\ket{2}-\frac{\braket{2}{1}}{\braket{1}{1}}=\ket{2}$$
$$\ket{3'}=\ket{3}-\frac{\braket{3}{1}}{\braket{1}{1}}\cdot\ket{1}-\frac{\braket{3}{2}}{\braket{2}{2}}\cdot\ket{2}=\frac{1}{\sqrt{3}}\mqty(-i \\ 1 \\ 0)$$
The first two u-vectors are both normalized, but the last one is not. We can easily see that it has length $\sqrt{2}$, so the normalized version of $\ket{3'}$ is 
$$\ket{3'}=\frac{1}{\sqrt{2}}\mqty(-i \\ 1 \\ 0)$$
The next step is to construct the matrix $H_{i'j'}$, but since $\ket{1}=\ket{1'}$ and $\ket{2}=\ket{2'}$, we only have to calculate the products which include $\ket{3'}$ on new:
$$\mel*{1'}{\hat{H}}{3'}=0$$
$$\mel*{2'}{\hat{H}}{3'}=0$$
$$\mel*{3'}{\hat{H}}{3'}=\frac{1}{2}$$
$$\mel*{3'}{\hat{H}}{2'}=0$$
$$\mel*{3'}{\hat{H}}{1'}=0$$
The rest elements of the matrix $H_{i'j'}$ are equal to the elements of $H_{ij}$, and we get this matrix:
$$H_{i'j'}=\mqty(\frac{3}{2} & 0 & 0 \\ 0 & \frac{1}{2} & 0 \\ 0 & 0 & \frac{1}{2})$$
There is easy to see that this matrix is diagonal.\\\\
NB: We could solve this problem some easier by noticing that we could use the Kronecker-Delta since $\ket{i'}$ is eigenvalue for $\hat{H}$. 

\newpage
\section*{Exercise 2.2}
\subsection*{a)}
In this exercise I will find the hermitian conjugate of $i$, $x^2$ and $\frac{d}{dx}$.\par\vspace{3mm}
I will start with $\hat{P}=i$:
$$\mel*{\phi_1}{\hat{P}^\dagger}{\phi_2}=\mel*{\phi_2}{\hat{P}}{\phi_1}^*=\bigg[\int_{-\infty}^{\infty} \phi_2^*(x)i\phi_1(x)dx\bigg]^*=\int_{-\infty}^{\infty} \phi_1^*(x)(-i)\phi_2(x)dx=-\mel*{\phi_1}{\hat{P}}{\phi_2}$$
So the hermitian conjugate of $\hat{P}$ is $\underline{-i}$.\par\vspace{6mm}
The next operator is $\hat{P}=x^2$:
$$\mel*{\phi_1}{\hat{P}^\dagger}{\phi_2}=\mel*{\phi_2}{\hat{P}}{\phi_1}^*=\bigg[\int_{-\infty}^{\infty} \phi_2^*(x)x^2\phi_1(x)dx\bigg]^*=\int_{-\infty}^{\infty} \phi_1^*(x)x^2\phi_2(x)dx=\mel*{\phi_1}{\hat{P}}{\phi_2}$$
Here we can see that $\underline{\hat{P}^\dagger=\hat{P}=x^2}$ \par\vspace{6mm}
Finally I will solve for $\hat{P}=\frac{d}{dx}:$
$$\mel*{\phi_1}{\hat{P}^\dagger}{\phi_2}=\mel*{\phi_2}{\hat{P}}{\phi_1}^*=\bigg[\int_{-\infty}^{\infty} \phi_2^*(x)\frac{d}{dx}\phi_1(x)dx\bigg]^*$$
Now I have to do partial integration. I choose $u=\phi_2^*(x)$ and $v'=\frac{d}{dx}\phi_1(x)$.
$$\Rightarrow\bigg[\phi_2(x)\phi_1^*(x)\bigg]_{-\infty}^{\infty}-\bigg[\int_{-\infty}^{\infty} \frac{d}{dx}\phi_2^*(x)\phi_1(x)dx\bigg]^*=-\int_{-\infty}^{\infty} \phi_1^*(x)\frac{d}{dx}\phi_2(x)dx=-\mel*{\phi_1}{\hat{P}}{\phi_2}$$
The hermitian conjugate in this case is $\underline{-\frac{d}{dx}}$.


\subsection*{b)}
In this exercise I will find the hermitian conjugate of the composite operator $\hat{H}=\hat{K}\hat{L}$. I will do this with braket-calculations:
$$\mel*{u}{\hat{H}^\dagger}{u}=\mel*{u}{\hat{H}}{u}^*=\mel*{u}{\hat{K}\hat{L}}{u}$$
$$=\sum_{n} \mel*{u}{\hat{K}}{n}^* \mel*{n}{\hat{L}}{u}^*$$
Note: I can do this since $I=\sum_{n}\ket{n}\bra{n}$
$$\Rightarrow \sum_{n} \mel*{n}{\hat{K}^\dagger}{u} \mel*{u}{\hat{L}^\dagger}{n}$$
$$=\sum_{n} \mel*{u}{\hat{L}^\dagger}{n} \mel*{n}{\hat{K}^\dagger}{u}=\mel*{u}{\hat{L}^\dagger\hat{K}^\dagger}{u}$$
So the hermitian conjugate of $\hat{K}\hat{L}$ is $\hat{L}^\dagger\hat{K}^\dagger$.
$$\underline{(\hat{K}\hat{L})^\dagger=\hat{L}^\dagger\hat{K}^\dagger}$$

\subsection*{c)}
In this exercise I will show that $\mel*{\lambda}{\hat{K}\hat{L}}{g}=\mel*{\lambda}{\hat{L}}{g}\lambda$:
$$\mel*{\lambda}{\hat{K}\hat{L}}{g}=\mel*{g}{(\hat{K}\hat{L})^\dagger}{\lambda}^*=\mel*{g}{\hat{L}^\dagger\hat{K}^\dagger}{\lambda}^*=\mel*{g}{\hat{L}^\dagger\hat{K}}{\lambda}^*$$
Comment: Here I first used the result from the previous exercise, and then I used that a hermitian operator always has the property that the eigenvalue is real.
$$\Rightarrow\mel*{g}{\hat{L}^\dagger\lambda}{\lambda}^*=\mel*{\lambda}{\hat{L}\lambda}{g}=\underline{\mel*{\lambda}{\hat{L}}{g}\lambda}$$
This is what I was supposed to show.
\end{document}