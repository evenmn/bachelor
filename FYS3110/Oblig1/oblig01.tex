\documentclass{scrartcl}
\usepackage{physics}

\title{FYS3110\\Quantum Mechanics}
\subtitle{Oblig 01}
\author{Even Marius Nordhagen}
\date{\today}

\begin{document}
\maketitle
\newpage
\section{Introduction}
The purpose of this problem set is to repeat what we have learned in the linear algebra courses, and to be familiar with the Dirac notation.

\section{Exercise 1}
We start with this ket:\par
\begin{equation}
\ket{\psi}=c(\sqrt{5}\ket{0})+i\ket{1}
\end{equation}
\paragraph{a)}
\begin{equation*}
\bra{\psi}\ket{\phi}=\bra{\phi}\ket{\psi}^*=c^*(\sqrt{5}\bra{\phi}\ket{0}^*-i\bra{\phi}\ket{1}^*)
\end{equation*}
\begin{equation*}
=c^*(\sqrt{5}\bra{0}\ket{\phi}-i\bra{1}\ket{\phi})
\end{equation*}
\begin{equation*}
=c^*(\sqrt{5}\bra{0}-i\bra{1})\ket{\phi}=\bra{\psi}\ket{\phi}
\end{equation*}
So $\bra{\psi}$ has to be
\begin{equation}
\bra{\psi}=c^*(\sqrt{5}\bra{0}-i\bra{1})
\end{equation}
I want to find the value of $c$ when the hypotenuse is fixed to $1$. This is found by Pythagoras:
\begin{equation*}
1=\sqrt{c^2\sqrt{5}^2+c^2}=c\sqrt{6}
\end{equation*}
\begin{equation}
c=\frac{1}{\sqrt{6}}
\end{equation}

\paragraph{b)}
We have that
$$\ket{0}\simeq\begin{pmatrix}1\\0\end{pmatrix},\quad \ket{1}\simeq\begin{pmatrix}0\\1\end{pmatrix}$$

I want to write $\ket{\psi}$ as a matrix, and from Equation (1) I see that
\begin{equation}
\ket{\psi}=c(\sqrt{5}\begin{pmatrix}1\\0\end{pmatrix}+i\begin{pmatrix}0\\1\end{pmatrix})
=c\begin{pmatrix}\sqrt{5}\\i\end{pmatrix}
\end{equation}
I also want to find the $\hat{A}$-matrix, and for that I need the equations
$$\hat{A}\ket{0}=-i\ket{1},\quad\hat{A}\ket{1}=i\ket{0}$$
On matrix form I then have two equations
$$\begin{pmatrix}a_{11}&a_{12}\\a_{21}&a_{22}\end{pmatrix}\begin{pmatrix}1\\0\end{pmatrix}
=\begin{pmatrix}0\\-i\end{pmatrix},\quad\begin{pmatrix}a_{11}&a_{12}\\a_{21}&a_{22}\end{pmatrix}\begin{pmatrix}0\\1\end{pmatrix}
=\begin{pmatrix}i\\0\end{pmatrix}$$
Which leads to $a_{11}=0,\;a_{12}=i,\;a_{21}=-i,\;a_{22}=0$, and
\begin{equation}
\hat{A}=\begin{pmatrix}0&i\\-i&0\end{pmatrix}
\end{equation}

\paragraph{c)}
First I want to compute $\expval{\hat{A}}{\psi}$ by using the representation in the previous sub exercise, but then I need to find $\bra{\psi}$ on matrix form. The question is, what are $\bra{0}$ and $\bra{1}$ represented by?\par \vspace{3mm}
This is not so hard, on the basis of normalization we can write
$$\braket{0}{0}=\bra{0}\begin{pmatrix}1\\0\end{pmatrix}=1$$
$$\braket{1}{1}=\bra{1}\begin{pmatrix}0\\1\end{pmatrix}=1$$
which tell us that 
$$\bra{0}\simeq(1,0),\quad\bra{1}\simeq(0,1)$$
If we insert these in Equation (2), we easily find that
$$\bra{\psi}=c^*(\sqrt{5},-i)$$
and finally we can begin computing $\expval{\hat{A}}{\psi}$:
\begin{equation*}
\expval{\hat{A}}{\psi}=c^*(\sqrt{5},-i)\begin{pmatrix}0&i\\-i&0\end{pmatrix}
c\begin{pmatrix}\sqrt{5}\\i\end{pmatrix}=c^*c(-\sqrt{5}-\sqrt{5})
\end{equation*}
$$\expval{\hat{A}}{\psi}=\underline{-2\sqrt{5}c^2}$$\par \vspace{5mm}
Now I want to compute $\expval{\hat{A}}{\psi}$ directly from the definitions:
$$\expval{\hat{A}}{\psi}=\bra{\psi}\hat{A}|c(\sqrt{5}\ket{0}+i\ket{1})$$
$$=\bra{\psi}c(\sqrt{5}\hat{A}\ket{0}+i\hat{A}\ket{1})$$
$$=\bra{\psi}c(-\sqrt{5}i\ket{1}+i^2\ket{0})$$
I also need to use my expression of $\bra{\psi}$ from Equation 2:
$$\Rightarrow c^*(\sqrt{5}\bra{0}-i\bra{1})c(-\sqrt{5}i\ket{1}-\ket{0})$$
$$=c^*c(-5i\braket{0}{1}-\sqrt{5}\braket{0}{0}-\sqrt{5}\braket{1}{1}+i\braket{1}{0})$$
$$\expval{\hat{A}}{\psi}=\underline{-2\sqrt{5}c^2}$$
\vspace{5mm}
Mark: The inner products $\braket{i}{j}$ are valued by
\begin{equation}
\braket{i}{j} =
    \begin{cases}
            1, &         \text{if } i=j,\\
            0, &         \text{if } i\neq j.
    \end{cases}
\end{equation}
and is called the Kronecker delta $\delta_{ij}$.
 

\section{Exercise 2}
$$U=\begin{pmatrix}a&b\\c&d\end{pmatrix}$$
\paragraph{a)}
$$U^T=\begin{pmatrix}a&c\\b&d\end{pmatrix},\quad
U^\dagger=\begin{pmatrix}a^*&c^*\\b^*&d^*\end{pmatrix}$$
\paragraph{b)}
A hermitian matrix is equal to its own conjugate transposed. With other words, if $U$ is hermitian, it has to satisfy
$$U=\begin{pmatrix}a&b\\c&d\end{pmatrix}=\begin{pmatrix}a^*&c^*\\b^*&d^*\end{pmatrix}$$
This give us this set of equations:
\renewcommand{\labelenumi}{\roman{enumi}}
\begin{enumerate}
\item .\quad $a=a^*$
\item .\quad $d=d^*$
\item .\quad $b=c^*$
\item .\quad $c=b^*$
\end{enumerate}
Equation 1 and 2 tell us that $a$ and $d$ have to be real numbers, and equation 3 and 4 tell us that $b$ and $c$ have to be in the same number category (both have to be either real or complex).
\paragraph{c)}
I compute the the eigenvalues by the standard formula
\begin{equation}
det(U-\lambda I)
\end{equation}
$$=\mqty|a-\lambda&b\\c&d-\lambda|=(a-\lambda)(d-\lambda)-bc
=\lambda^2-(a+d)\lambda+(ad-bc)=0$$
To find the roots for this characteristic polynomial, I have to use the famous ABC-formula:
\begin{equation}
\lambda=\frac{-B\pm\sqrt{B^2-4AC}}{2A}
\end{equation}
In our case $A=1$, $B=-(a+d)$ and $C=(ad-bc)$. By inserting this, we will get that
\begin{equation}
\lambda=\frac{(a+d)\pm\sqrt{(a+d)^2-4(ad-bc)}}{2}=\frac{a+d\pm\sqrt{(a-d)^2+4bc}}{2}
\end{equation}
I can not take this longer, but this is good enough for proving that the eigenvalues always are real when $U$ is hermitian. There are two ways that $\lambda$ can be complex. 
A possibility is when one or more of the constants are complex. In the previous sub exercise we saw that $a$ and $d$ must be real, and since $4bc=4c^*c \in \Re $, we should not worry about this case.
\par \vspace{3mm}
An other way that $\lambda$ may could be complex, is if the square root is negative. This is neither possible, because both of the terms must me positive. So $\lambda$ has to be a real number.
\paragraph{d)}
We call a matrix unitary when it has the properties 
\begin{equation*}
U^*U=UU^*=I,\quad U^\dagger U=UU^\dagger=I
\end{equation*}
and we have already seen that a matrix 
$$U=\mqty(a & b \\ c & d)$$
is hermitian when $a$ and $d$ are real numbers, and $b$ and $c$ are in the same category. A hermitian unitary matrix therefore has to satisfy the matrix equation
$$\mqty(a & b \\ c & d)\mqty(a & b \\ c & d)=\mqty(aa + bc & ab + bd \\ ac + cd & bc + dd)=\mqty(\imat{2})$$
From this we can find a set with four independent equations:
\renewcommand{\labelenumi}{\roman{enumi}}
\begin{enumerate}
\item .\quad $a^2+bc=1$
\item .\quad $ab+bd=0$
\item .\quad $ac+cd=0$
\item .\quad $bc+d^2=1$
\end{enumerate}
Of course I could solve this set of equation by row operations on the matrix, but I think it goes just as fast to solve the equations on the "primary school way". Anyway we get two different solutions that depends on how we choose $b$:
\renewcommand{\labelenumi}{\roman{enumi}}
\begin{enumerate}
\item .\quad $b=0:$
We can see that this is a solution that satisfies equation ii. and iii., and we have
$$a^2=d^2=1$$ 
\item .\quad $b\neq 0:$
This is of course also a possible solution which gives us that 
$$a=\pm\sqrt{1-b^2},\quad d=\mp\sqrt{1-b^2},\quad a=-d$$
Mark: Since $a$ and $d$ have to be positive, we have to choose a $b$ in the interval $b\in[-1,1]$.
\end{enumerate}
\paragraph{e)}
After we have choose $b$, we can easily find the eigenvalues by inserting into Equation (9). I'll solve it for the two solutions, and I start with $b=0$:
\renewcommand{\labelenumi}{\Roman{enumi}}
\begin{enumerate}
\item $$a=\pm1 \quad b=0\quad c=0\quad d=\pm1$$
$$\lambda=\frac{\pm1\pm1\pm(\pm1\pm1)}{2}=\pm1$$
\item $$a=\pm\sqrt(1-b^2)\quad b\neq0 \quad c=b \quad d=-a$$
$$\lambda=\pm\frac{\sqrt{4(1-b^2)+4b^2}}{2}=\pm1$$
So the conclusion is that the eigenvalues have to be $\pm$!
\end{enumerate}
\section{Exercise 3}
We start with the following statements:
$$\hat{H}\ket{\psi}=g\ket{\phi},\quad \hat{H}\ket{\phi}=g^*\ket{\psi},\quad \hat{H}\ket{\gamma_n}=0$$
If the operator $\hat{H}$ is hermitian, it has to satisfy $\hat{H}^\dagger=\hat{H}$. I will try to find which conditions $\ket{\phi}$ and $\ket{\psi}$ must satisfy if $\hat{H}$ is hermitian:
$$\mel*{\phi}{\hat{H}}{\psi}=\mel*{\phi}{g}{\phi}=g\braket{\phi}{\psi}=g$$
$$\mel*{\phi}{\hat{H^\dagger}}{\psi}=\mel*{\psi}{\hat{H}}{\phi}=g\braket{\psi}{\psi}^*=g\braket{\psi}{\psi}=g$$
So $\hat{H}$ is hermitian if $\psi$ and $\phi$ are normalized.
\section{Comment}
I decided try to write this delivery in English just to improve my English skills. Please tell me if something is terrible, I'm sure I have a lot to learn about writing physic reports in English. 
\end{document}
