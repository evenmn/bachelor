\documentclass{scrartcl}
\usepackage{physics}   % Matrices and Dirac-notation
\usepackage{amsmath}   % Binear equations
\usepackage{booktabs}  % Tabs
\usepackage{graphicx}  % Pictures/figures
\usepackage{listings}  % Source code
\usepackage{color}     % Colors
\usepackage{mdframed}  % Frames

\definecolor{dkgreen}{rgb}{0,0.6,0}
\definecolor{gray}{rgb}{0.5,0.5,0.5}
\definecolor{mauve}{rgb}{0.58,0,0.82}

%Defining source code
\lstset{frame=tb,
  language=Python,
  aboveskip=3mm,
  belowskip=3mm,
  showstringspaces=false,
  columns=flexible,
  basicstyle={\small\ttfamily},
  numbers=none,
  numberstyle=\tiny\color{gray},
  keywordstyle=\color{blue},
  commentstyle=\color{dkgreen},
  stringstyle=\color{mauve},
  breaklines=true,
  breakatwhitespace=true,
  tabsize=3
}
\makeatletter
\renewcommand*\env@matrix[1][*\c@MaxMatrixCols c]{%
  \hskip -\arraycolsep
  \let\@ifnextchar\new@ifnextchar
  \array{#1}}
\makeatother
\begin{document}
\begin{titlepage}
	\centering
	{\scshape\LARGE $\star\star\star\,\star $  \par}
	\vspace{4cm}
	{\scshape\huge FYS3110 - Quantum Mechanics  \par}
	\vspace{1cm}
	{\scshape\Large Oblig 04\par}
	\vspace{2cm}
	{\Large\itshape Even Marius Nordhagen\par}
	\vfill
	{\large \today\par}
\end{titlepage}

\section*{4.1}
\subsection*{a)}
If we add $\hat{a}^\dagger$ to $\hat{a}$ and multiplying with $\sqrt{\frac{\hbar}{2m\omega}}$ on both sides, we get an expression for the position operator:
\begin{equation}
\hat{X}=\sqrt{\frac{\hbar}{2m\omega}}\bigg(\hat{a}^\dagger \hat{a}\bigg)
\end{equation}
I will also use that
$$\hat{a}^\dagger \ket{n}=\sqrt{n+1}\ket{n+1},\quad\hat{a}\ket{n}=\sqrt{n}\ket{n-1}, \hat{a}\ket{0}=0$$
In general we have
$$\bra{n'}\hat{X}\ket{n}=\bra{n'}\Bigg(\sqrt{\frac{\hbar}{2m\omega}}\bigg(\hat{a}^\dagger\ket{n}+\hat{a}\ket{n}\bigg)\Bigg)$$
$$=\bra{n'}\Bigg(\sqrt{\frac{\hbar}{2m\omega}}\bigg(\sqrt{n+1}\ket{n+1}+\sqrt{n}\ket{n-1}\bigg)\Bigg)=\sqrt{\frac{\hbar}{2m\omega}}\bigg(\sqrt{n+1}\braket{n'}{n+1}+\sqrt{n}\braket{n'}{n-1}\bigg)$$
Here is the famous Kronecker-delta, so the only cases where we do not get zero, is when $$n'=n+1:\quad \mel{n'}{\hat{X}}{n+1}=1$$
$$n'+1=n:\quad \mel{n'+1}{\hat{X}}{n}=1$$
Therefore we get 
$$n'=n+1:\bra{n'}\hat{X}\ket{n}=\quad\sqrt{n+1}\sqrt{\frac{\hbar}{2m\omega}}$$
$$n'+1=n:\bra{n'}\hat{X}\ket{n}=\quad\sqrt{n}\sqrt{\frac{\hbar}{2m\omega}}$$
And we obtain the matrix:
\begin{equation}
\mqty[0&\sqrt{\frac{\hbar}{m\omega}}&\hdots&0&0\\
      \sqrt{\frac{\hbar}{m\omega}}&0&\hdots&0&0\\
      \vdots&\vdots&\ddots&\vdots&\vdots\\
      0&0&\hdots&0&\sqrt{\frac{(n+1)\hbar}{2m\omega}}\\
      0&0&\hdots&\sqrt{\frac{(n)\hbar}{2m\omega}}&0]
\end{equation}
\subsection*{b)}
An eigenstate can be expressed as an infinity number of constants:
\begin{equation}
\ket{\psi(0)}=\sum_{n=0}^\infty C_n\ket{n}
\end{equation}
In general we have that 
\begin{equation}
C_n=\braket{m}{\psi(0)}
\end{equation}
Which have to be normalized, so
\begin{equation}
\sqrt{\sum_{n=0}^\infty C_n}=1
\end{equation}
To find the time-depended state, we need to multiply with the Propagator, like this:
\begin{equation*}
\hat{U}\ket{\psi(t)}=\hat{U}\sum_{n=0}^\infty=\hat{U}C_ne^{-iE_nt/\hbar}\ket{E_n}
\end{equation*}
\begin{equation}
\ket{\psi(t)}=\sum_{n=0}^\infty C_ne^{-iE_nt/\hbar}\ket{n}
\end{equation}

\subsection*{c)}
I will start to find the expectation value of the Hamiltonian (expectation value for energy). For that I need to use the Schrodinger equation from the description of Exercise $a)$:
\begin{equation}
\hat{H}\ket{n}=\hbar\omega(n+1/2)\ket{n}
\end{equation}
and the expression for the time-dependent states found in the previous exercise (\textit{see Equation (6)}).\par\vspace{3mm}
$$\hat{H}\ket{\psi(t)}=\sum_{n=0}^\infty C_ne^{-iE_nt/\hbar}\hbar\omega(n+1/2)\ket{n}$$
$$\mel{\psi(t)}{\hat{H}}{\psi(t)}=\sum_{n=0}^\infty C_n^*e^{iE_nt/\hbar}\bra{n}\sum_{n=0}^\infty C_ne^{-iE_nt/\hbar}\hbar\omega(n+1/2)\ket{n}$$
$$=\sum_{n'=0}^\infty \sum_{n=0}^\infty C_{n'}^*C_ne^{-(1-1)iE_nt/\hbar}\hbar\omega(n+1/2)\braket{n'}{n}$$
$$=\underline{\hbar\omega\sum_{n=0}^\infty |C_n|^2(n+1/2)}$$
Where I have used the Kronecker-delta and $e^0=1$.
The next step is to calculate the expectation value of position, and again I need the inverse of the operators $\hat{a}^\dagger$ and $\hat{a}$ from Equation (1). Then I will find that
$$\mel{\psi(t)}{\hat{X}}{\psi(t)}=\sqrt{\frac{\hbar}{2m\omega}}\sum_{n'=0}^\infty \sum_{n=0}^\infty C_{n'}^*C_n e^{iE_{n'}t/\hbar}e^{-iE_nt/\hbar}\bra{n'}\bigg(\sqrt{n+1}\ket{n+1}+\sqrt{n}\ket{n-1}\bigg)$$
So far the exponential functions have disappeared since we multiply who exponential functons with equal exponents just with opposite signs. This is not the situation here, so I will use the relation $E_n=\hbar\omega(n+1/2)$:
$$\sqrt{\frac{\hbar}{2m\omega}}\sum_{n'=0}^\infty \sum_{n=0}^\infty C_{n'}^*C_n e^{i\omega t}\bigg(\sqrt{n+1}\braket{n'}{n+1}+\sqrt{n}\braket{n'}{n-1}\bigg)$$
The first inner product will be be non-zero only when $n'=n+1$ and the second one will be non-zero when $n'=n-1$, so in the first term I can replace  $n'$ with $n+1$ and in the second term I can replace $n'$ with $n-1$. Finally we obtain 
\begin{equation}
\sqrt{\frac{\hbar}{2m\omega}}\sum_{n=0}^\infty\bigg[C_{n+1}^*C_ne^{i\omega t}\sqrt{n+1}+C_{n-1}^*C_ne^{-i\omega t}\sqrt{n}\bigg]
\end{equation}

\subsection*{d)}
Now I will use the expression above and replace the $C_n$ with
\begin{equation}
C_n=C_0\frac{\alpha^2}{\sqrt{n!}}
\end{equation}
Where $C_n$ are positive real numbers:
\begin{equation}
C_0^2\sqrt{\frac{\hbar}{2m\omega}}\sum_{n=0}^\infty \bigg[\sqrt{n+1}\frac{\alpha^{n+1}}{\sqrt{(n+1)!}}frac{\alpha^n}{\sqrt{n!}}e^{i\omega t}+\sqrt{n}\frac{\alpha^{n-1}}{\sqrt{(n-1)!}}frac{\alpha^n}{\sqrt{n!}}e^{-i\omega t}\bigg]
\end{equation}
I will now define $m=n-1$ and split up to two sums.
$$=C_0^2\sqrt{\frac{\hbar}{2m\omega}}\alpha\bigg[\sum_{n=0}^\infty \frac{\alpha^2n}{n!}e^{i\omega t}+\sum_{m=0}^\infty \frac{\alpha^2m}{m!}e^{-i\omega t}\bigg]$$
From Rottmann we have the general formula:
\begin{equation}
e^x=\sum_n \frac{x^n}{n!}
\end{equation}
Which gives us
$$\Rightarrow C_0^2 \sqrt{\frac{\hbar}{2m\omega}}\alpha e^{\alpha^2}\bigg(e^{i\omega t}+e^{-i\omega t}\bigg)$$
\begin{equation}
=\underline{\alpha \sqrt{\frac{\hbar}{2m\omega}} cos(\omega t)}
\end{equation}
Where I have used that
\begin{equation}
C_0=e^{-\alpha^2}
\end{equation}
\begin{equation}
2cos(\omega t)=e^{i\omega t}+e^{-i\omega t}
\end{equation}

\section*{4.2}
\subsection*{a)}
In this exercise I will show that the Schrodinger exuation for a harmonic oscillator in three dimensions can be separated in to three one dimensional harmonic oscillators. \par\vspace{3mm}
We already know that the Laplace operator can be split up:
\begin{equation}
\nabla^2=\bigg(\frac{\partial^2}{\partial x^2}+\frac{\partial^2}{\partial y^2}+\frac{\partial^2}{\partial z^2}\bigg)
\end{equation}
We also know that the potential can be separated since $\vec{r}^2=x^2+y^2+x^2$.
Then we have
\begin{equation}
-\frac{\hbar^2}{2m}\bigg(\frac{\partial^2}{\partial x^2}+\frac{\partial^2}{\partial y^2}+\frac{\partial^2}{\partial z^2}\bigg)\Psi+\frac{1}{2}m\omega^2(x^2+y^2+x^2)\Psi=E_n\Psi
\end{equation}
If we assume that the energy can be separated, we can split the Schrodinger equation into three parts. We will then obtain energies from each of the parts, and as we have seen before the energies are quantized and are given by $E_{n_i}=\hbar\omega(n_i+1/2)$. The total $E_n$ in three dimensions is therefore
\begin{equation}
E_n=E_x+E_y+E_z=\hbar\omega(n_x+n_y+n_z+1/2+1/2+1/2)=\hbar\omega(n+3/2)
\end{equation} 
where $n=n_x+n_y+n_z$.

\subsection*{b)}
An one dimensional harmonic oscillator (HO) will not have degenerated energies, but a three dimensional HO actually will. In the previous exercise we saw that 
\begin{equation}
n=n_x+n_y+n_z
\end{equation}
If we choose for instance $n_x$, we have
$$n_y+n_z=n-n_x$$
Where all variables need to be non-negative integers. Therefore the $n_y$ can be in the interval $n_y\in[0,n-n_x]$ for a fixed $n_x$ and the degeneracy for $n_y$ and $n_z$ is $n-n_x+1$.\par\vspace{3mm}
$n_x$ can range from 0 to $n$, so the total degeneracy is given by the sum
\begin{equation}
d(n)=\sum_{n_x=0}^n (n-n_x+1)=(n+1)\sum_{n_x=0}^n 1-\sum_{n_x=0}^n n_x
\end{equation}
The first sum is simply (n+1), but the last one is more difficult. From Rottmann we have that
\begin{equation}
\sum_{x=0}^n x = \frac{n(n+1)}{2}
\end{equation}
\begin{equation}
d(n)=(n+1)^2-\frac{1}{2}n(n+1)=\frac{1}{2}(n+1)(n+2)
\end{equation}

\end{document}